<<<<<<< HEAD
Relato de experiência da execução do trabalho;

    O trabalho nos permitiu colocar em execução grande parte do conteúdo aprendido em sala de aula
com um projeto de atual demanda no mundo real, unindo assim o ambiente acadêmico com o ambiente profissional,
teoria e prática.
Inicialmente, foi proposto que a escolha da metodologia fosse feita usando bases científicas, e não
intuitivas, como estávamos acostumados a fazer, isso nos proporcionou, atráves do estudo para conhecermos
mais profundamente algumas metodologias de desenvolvimento, conhecimento para podermos realizar a melhor
escolha diante do nosso contexto. E essa escolha também nos forneceu ainda mais expertise,
especialmente sobre os aspectos individuais da metodologia de desenvolvimento escolhida por nós.
A frase "Podemos perceber que era possível aplicar o Ágil em projetos grandes, através do SAFe"
expressa um entendimento claro de todos os membros do grupo, e foi obtido atráves da execução do trabalho
e dos estudos aprofundados relacionados a nossa escolha.
    Tivemos oportunidade também de conhecer ferramentas de gerenciamento de requisitos, e aprender a usar
aquela selecionada por nós como a mais eficaz.
    Podemos ter mais contato com a parte de ER, aprendendo assim, na prática, a melhor forma de diálogo
com o cliente, análisar problemas, propor soluções, quantificar requisitos, e principalmente,
entender como funciona um processo de desenvolvimento de software e ser capaz de discernir oque deve
e oque não deve estar presente em um processo, enfim, habilidades indispensáveis
para qualquer bom engenheiro de software.
    Concluímos assim, que o trabalho foi uma experiência positiva, que nos permitiu adquirir muitos
conhecimentos práticos dentro da área de ER, e que sem ele, talvez a teoria aplicada em sala de aula
ficaria um pouco obsoleta.
=======
\chapter{Relatos}

\section{Relato de experiência da execução do trabalho}
  Aprender mais sobre a metodologia escolhida, "Percebemos que era possível aplicar o ágil em projetos grandes, através de frameworks como o SAFe".

  Conhecemos mais sobre ferramentas de gerenciamento de software.

  Aprender a discernir metodologias mais eficazes para determinado projeto.
  mais conhecimento na área de engenharia de requisitos.

  Aprender a discernir de forma mais eficaz oque deve estar presente e oque não deve dentro de um Processo de desenvolvimento.

  Ter uma experiência de desenvolvimento de software unindo o ambiente acadêmico a atuais demandas do mundo real.
>>>>>>> 89dc155592d053c9eea885a66f6fe97ffaa37cc2

  Aprender a definir problemas.

\section{Relato de experiência da disciplina de ER}

Esta disciplina nos trouxe um novo olhar sobre a engenharia de software,
nosso contato prévio era bem restrito a área de desenvolvimento, e, de forma intuitiva, menosprezamos os esforços
que seriam dedicados na parte dos requisitos, e queríamos prosseguir para o desenvolvimento o rápido possível,
afinal, este era o nosso território conhecido, nossa zona de conforto.
De forma mais clara, podemos aprender sobre o processo de Engenharia de Requisitos dentro das duas metodologias
de desenvolvimento mais usadas atualmente, Ágil e UP.
Podemos ver também características de um bom requisito, e que manter sua rastreabilidade, horizontal e vertical,
é de suma importância, tanto na parte da ER, quanto na parte de implementação, visto que poder linkar o código
ao requisito a ele correspondente de forma rápida e eficaz é uma grande vantagem em qualquer processo.

Aprendemos também técnicas de elicitação, técnicas e formas de análise dos problemas do usuário,
formas de entender as necessidades dos clientes e, baseados nisso, propor uma solução, que, de forma esperada,
deve ser aquilo que o cliente precisa, e não necessáriamente aquilo que o cliente quer.
Podemos também conhecer muitos problemas dentro da ER, que podem estar presentes em todas as etapas do processo.

Outro aprendizado extremamente importante foi o relacionado a modelos de maturidade, CMMI e MPS-Br, de forma pessoal,
alguns integrantes do nosso grupo nem sabiam oque eram modelos de maturidade, ou como era realizado o processo de
avaliação de qualidade de um software.
<<<<<<< HEAD
=======

\begin{itemize}
\item Aprender mais e mais profundamente sobre Metodologias de Desenvolvimento, e o processo de ER dentro de cada uma delas.
\item Aprender sobre técnicas de elicitação e como entender oque o cliente precisa
\item Conhecer atuais problemas no ambiente de ER
  Aprender sobre modelos de maturidades

\end{itemize}
>>>>>>> 89dc155592d053c9eea885a66f6fe97ffaa37cc2
