Relato de experiência da execução do trabalho;
  Aprender mais sobre a metodologia escolhida, "Percebemos que era possível aplicar o ágil em projetos grandes, através do SAFe"
  Conhecemos mais sobre ferramentas de gerenciamento de software
  Aprender a discernir metodologias mais eficazes para determinado projeto.
  mais conhecimento na área de engenharia de requisitos.
  Aprender a discernir de forma mais eficaz oque deve estar presente e oque não deve dentro de um Processo de desenvolvimento.
  Ter uma experiência de desenvolvimento de software unindo o ambiente acadêmico a atuais demandas do mundo real.
  Aprender a definir problemas

Relato de experiência da disciplina de ER

Esta disciplina nos trouxe um novo olhar sobre a engenharia de software,
nosso contato prévio era bem restrito a área de desenvolvimento, e, de forma intuitiva, menosprezámos os esforços
que seriam dedicados na parte dos requisitos, e queriamos prosseguir para o desenvolvimento o rápido possível,
afinal, este era o nosso território conhecido, nossa zona de conforto.
De forma mais clara, podemos aprender sobre o processo de Engenharia de Requisitos dentro das duas metodologias
de desenvolvimento mais usadas atualmente, Ágil e UP.
Podemos ver também características de um bom requisito, e que manter sua rastreabilidade, horizontal e vertical,
é de suma importância, tanto na parte da ER, quanto na parte de implementação, visto que poder linkar o código
ao requisito a ele correspondente de forma rápida e eficaz é uma grande vantagem em qualquer processo.
Aprendemos também técnicas de elicitação, técnicas e formas de análise dos problemas do usuário,
formas de entender as necessidades dos clientes e, baseados nisso, propor uma solução, que, de forma esperada,
deve ser aquilo que o cliente precisa, e não necessáriamente aquilo que o cliente quer.
Podemos também conhecer muitos problemas dentro da ER, que podem estar presentes em todas as etapas do processo.
Outro aprendizado extremamente importante foi o relacionado a modelos de maturidade, CMMI e MPS-Br, de forma pessoal,
alguns integrantes do nosso grupo nem sabiam oque eram modelos de maturidade, ou como era realizado o processo de
avaliação de qualidade de um software.

 -  Aprender mais e mais profundamente sobre Metodologias de Desenvolvimento, e o processo de ER dentro de cada uma delas.
 -  Aprender sobre técnicas de elicitação e como entender oque o cliente precisa
 - Conhecer atuais problemas no ambiente de ER
  Aprender sobre modelos de maturidades
