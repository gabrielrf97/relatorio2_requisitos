\chapter{Processo de Engenharia de Requisitos}

Na elaboração da primeira etapa do trabalho, o grupo construiu um processo com atividades relacionadas a Engenharia de Requisitos, esse processo está representado abaixo, o capítulo irá abordar os detalhes dessas atividades realizadas de uma forma otimizada.

\begin{figure}[!htpb]
\centering	
\includegraphics[scale=0.45]{figuras/processo/modelo}
\caption{Modelo do processo a ser utilizado}
\end{figure}

\newpage

\section{Levantar temas estratégicos}

Segundo o SAFe, temas estratégicos são especicamente objetivos de negócio que conectam o nível de Portfólio com a estratégia de negócio da empresa, no caso do trabalho a equipe, então no contexto do trabalho ao se reunir com o cliente e entender o problema do mesmo, a equipe definiu um único tema estratégico que seria a Gerência de Estoque de forma Adequada.

\begin{itemize}
\item Gerência de Estoque de forma adequada: A proposta do grupo envolve fornecer um produto onde seria demandado de forma tecnológica e prática a gerência do estoque utilizando de uma aplicação para Desktop. Seriam também feitas medidas que ajudassem no controle de desperdício de estoque de produção, para que não haja gastos em excesso.
\end{itemize}

\section{Definir épicos}
Segundo o SAFe, épicos de negócio agregam valor de forma direta no projeto. Foram coletados dois épicos de negócio junto ao cliente, que são listados abaixo:

\begin{enumerate}
\item Informatizar Sistema de Controle de Estoque
	\begin{table}[htb]
    \begin{tabular}{|l|l|}
        \hline
        Para          & {\parbox{12cm}{A empresa e seus funcionários}}                                        \\ \hline
        Quem          & {\parbox{12cm}{controlam estoque      }}                                                \\ \hline
        O             & {\parbox{12cm}{Pydaria}}                                                                \\ 
        É             & {\parbox{12cm}{uma ferramenta para controlar estoque         }}                         \\ \hline
        Que           & {\parbox{12cm}{ajuda a controlar o estoque de uma maneira mais organizada e eficiente}} \\ \hline
        Diferente de  & {\parbox{12cm}{ Microsoft Excel       }}                                                 \\ 
        Nossa Solução & {\parbox{12cm}{faz esse controle de uma forma mais dinâmica.     }}                     \\ \hline
        \hline
    \end{tabular}
    \caption{Posicionamento do épico 2.}
    \end{table}

    \begin{table}[htb]
    \begin{tabular}{|l|l|}
        \hline
        Critério de Aceitação & {\parbox{12cm}{O produto tem que fornecer o controle de estoque de uma maneira simples e efetiva, de forma a facilitar o trabalho dos funcionários.}} \\ \hline
        No Escopo             & {\parbox{12cm}{Contabilização de estoque através dos produtos disponíveis para venda e produção. }} \\ \hline
        Fora do escopo        & {\parbox{12cm}{Sincronizar o sistema com algum código identificador nos produtos para identificar sua saída. }}                                       \\
        \hline
    \end{tabular}
    \caption{Escopo do épico 1.}
	\end{table}

\newpage

\item Informatizar sistema para controle de desperdício
	\begin{table}[htb]
    \begin{tabular}{|l|l|}
        \hline
        Para          & {\parbox{12cm}{A empresa e seus funcionários}} \\ \hline
        Quem          & {\parbox{12cm}{controlam estoque}}                                                      \\ \hline
        O             & {\parbox{12cm}{Pydaria}}                                                               \\ \hline
        É             & {\parbox{12cm}{uma ferramenta que dentre outras coisas auxilia os funcionários e proprietários da empresa controlar estoque}}                                \\ \hline
        Que           & {\parbox{12cm}{auxilia o controle de desperdício indicando a melhor tomada de decisão diante de reposição de estoque e produção.}} \\ \hline
        Diferente de  & {\parbox{12cm}{Microsoft Excel e controlar de forma empírica  }}                                                    \\ \hline
        Nossa Solução & {\parbox{12cm}{sugere a tomada de decisão de quantidade de produtos para a reposição ou quantidade de ingredientes para a produção de algum produto dependendo do treinamento de aprendizado de máquina baseado nos dados coletados do tempo do sistema em produção.}}                          \\
        \hline
    \end{tabular}
    \caption{Posicionamento do épico 1.}
    \end{table}


     \begin{table}[htb]
    \begin{tabular}{|l|l|}
        \hline
        Critério de Aceitação & {\parbox{12cm}{O software deve sugerir baseado em dados anteriores registrados pelos envolvidos no controle de estoque e na produção a quantidade de produtos na reposição de estoque e a quantidade de matéria prima envolvida na produção.}} \\ \hline
        No Escopo             & {\parbox{12cm}{Sugestão de quantidade de produtos para a reposição e matéria prima para produção}}                                                \\ 
        Fora do escopo        & {\parbox{12cm}{Sugestões baseadas em outros dados que não seja tempo.}}                             \\
        \hline
    \end{tabular}
    \caption{Escopo do épico 2.}
	\end{table}

\end{enumerate}

\section{Definir enablers dos épicos}

\section{Priorizar Épicos}

\section{Definir Features}

\section{Definir Enablers das Features}

\section{Construir o Visão}

\section{Priorizar Features}

\section{Definir o Roadmap}

\section{Planejar o PI}

\section{Desmembrar Features em Histórias de Usuários}

\section{Priorizar Histórias de Usuário}

\section{Planejar Iteração}

\section{Desenvolver Histórias de Usuários}

\section{Realizar revisão da iteração}
