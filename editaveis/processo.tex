\chapter{Processo de Engenharia de Requisitos}

Na elaboração da primeira etapa do trabalho, o grupo construiu um processo com atividades relacionadas a Engenharia de Requisitos, esse processo está representado abaixo, o capítulo irá abordar os detalhes dessas atividades realizadas de uma forma otimizada.

\begin{figure}[!htpb]
\centering	
\includegraphics[scale=0.35]{figuras/processo/modelo}
\caption{Modelo do processo a ser utilizado}
\end{figure}

\newpage

\section{Levantar temas estratégicos}

Segundo o SAFe, temas estratégicos são especicamente objetivos de negócio que conectam o nível de Portfólio com a estratégia de negócio da empresa, no caso do trabalho a equipe, então no contexto do trabalho ao se reunir com o cliente e entender o problema do mesmo, a equipe definiu um único tema estratégico que seria a Gerência de Estoque de forma Adequada.

\begin{itemize}
\item Gerência de Estoque de forma adequada: A proposta do grupo envolve fornecer um produto onde seria demandado de forma tecnológica e prática a gerência do estoque utilizando de uma aplicação para Desktop. Seriam também feitas medidas que ajudassem no controle de desperdício de estoque de produção, para que não haja gastos em excesso.
\end{itemize}

\section{Definir épicos}
Segundo o SAFe, épicos de negócio agregam valor de forma direta no projeto. Foram coletados dois épicos de negócio junto ao cliente, que são listados abaixo:


\subsection{Informatizar Sistema de Controle de Estoque}
	
	\begin{table}[htb]
    \begin{tabular}{|l|l|}
        \hline
        Para          & {\parbox{12cm}{A empresa e seus funcionários}}                                        \\ \hline
        Quem          & {\parbox{12cm}{controlam estoque      }}                                                \\ \hline
        O             & {\parbox{12cm}{Pydaria}}                                                                \\ 
        É             & {\parbox{12cm}{uma ferramenta para controlar estoque         }}                         \\ \hline
        Que           & {\parbox{12cm}{ajuda a controlar o estoque de uma maneira mais organizada e eficiente.  }} \\ \hline
        Diferente de  & {\parbox{12cm}{ Microsoft Excel       }}                                                 \\ \hline
        Nossa Solução & {\parbox{12cm}{faz esse controle de uma forma mais dinâmica.     }}                     \\
        \hline
    \end{tabular}
    \caption{Posicionamento do épico 1.}
    \end{table}

    \begin{table}[htb]
    \begin{tabular}{|l|l|}
        \hline
        Critério de Aceitação & {\parbox{12cm}{O produto tem que fornecer o controle de estoque de uma maneira simples e efetiva, de forma a facilitar o trabalho dos funcionários.}} \\ \hline
        No Escopo             & {\parbox{12cm}{Contabilização de estoque através dos produtos disponíveis para venda e produção. }} \\ \hline
        Fora do escopo        & {\parbox{12cm}{Sincronizar o sistema com algum código identificador nos produtos para identificar sua saída. }}                                       \\
        \hline
    \end{tabular}
    \caption{Escopo do épico 1.}
	\end{table}

\newpage

\subsection{Informatizar sistema para controle de desperdício}
	\begin{table}[htb]
    \begin{tabular}{|l|l|}
        \hline
        Para          & {\parbox{12cm}{A empresa e seus funcionários}} \\ \hline
        Quem          & {\parbox{12cm}{controlam estoque}}                                                      \\ \hline
        O             & {\parbox{12cm}{Pydaria}}                                                               \\ \hline
        É             & {\parbox{12cm}{uma ferramenta que dentre outras coisas auxilia os funcionários e proprietários da empresa controlar estoque}}                                \\ \hline
        Que           & {\parbox{12cm}{auxilia o controle de desperdício indicando a melhor tomada de decisão diante de reposição de estoque e produção.}} \\ \hline
        Diferente de  & {\parbox{12cm}{Microsoft Excel e controlar de forma empírica  }}                                                    \\ \hline
        Nossa Solução & {\parbox{12cm}{sugere a tomada de decisão de quantidade de produtos para a reposição ou quantidade de ingredientes para a produção de algum produto dependendo do treinamento de aprendizado de máquina baseado nos dados coletados do tempo do sistema em produção.}}                          \\
        \hline
    \end{tabular}
    \caption{Posicionamento do épico 2.}
    \end{table}


     \begin{table}[htb]
    \begin{tabular}{|l|l|}
        \hline
        Critério de Aceitação & {\parbox{12cm}{O software deve sugerir, baseado em dados anteriores registrados pelos envolvidos no controle de estoque e na produção, a quantidade de produtos na reposição de estoque e a quantidade de matéria prima envolvida na produção.}} \\ \hline
        No Escopo             & {\parbox{12cm}{Sugestão de quantidade de produtos para a reposição e matéria prima para produção}}                                                \\ \hline	
        Fora do escopo        & {\parbox{12cm}{Sugestões baseadas em outros dados que não seja tempo.}}                             \\
        \hline
    \end{tabular}
    \caption{Escopo do épico 2.}
	\end{table}


\section{Definir enablers dos épicos}

\section{Priorizar Épicos}
Foram levantados apenas dois épicos de negócio, e ambos foram priorizados para serem trabalhados.

\section{Definir Features}
Features é um serviço fornecido pelo sistema, de acordo com o SAFe. Seguindo o modelo de rastreabilidade as features precisam ter seus épicos, logo serão detalhadas as features de cada épico separadamente.

\subsection{Features do Épico: Informatizar Sistema de Controle de Estoque}

\begin{itemize}
\item Feature: Manter produtos
\begin{itemize}
\item Descrição: Essa feature envolve as operações de CRUD(Create, Read, Update, Delete) básicas da maioria dos softwares, aplicadas aos recursos gerenciados na empresa.
\item Benefícios: Fornecer melhor catalogação dos produtos, substituindo a forma empírica.
\end{itemize}

\item Feature: Analisar taxa de entrada/saída de produtos específicos
\begin{itemize}
	\item Descrição: Verificar se a taxa de saída corresponde a taxa de entrada.
	\item Benefícios: Permite controle do estoque desses produtos contra atos de furto.
\end{itemize}

\end{itemize}	

\subsection{Features do Épico: Informatizar sistema para controle de desperdício}

\begin{itemize}

\item Feature: Analisar o uso de matéria-prima
\begin{itemize}
	\item Descrição:
	\item Benefícios:
\end{itemize}

\item Feature: Informar sobre os resultados mensais
\begin{itemize}
	\item Descrição:
	\item Benefícios:
\end{itemize}

\end{itemize}

\section{Definir Enablers das Features}

\section{Construir o Visão}

Segundo o SAFe, o documento de Visão descreve uma visão futurística do produto a ser desenvolvido, refletindo o que os stakeholders e clientes necessitam, dando uma visão geral do que é e do que faz o produto.
O documento de visão do projeto segue em anexo ao relatório.


\section{Definir o Roadmap}
Segundo o SAFe, o Roadmap serve para informar as metas de releases do software em cronogramas aproximados.
O RoadMap segue em anexo ao relatório.

\section{Planejar o PI}
O planejamento do Product Increment consiste no encontro cara-a-cara da equipe para definir quantas iterações serão necessárias para implementar aquele incremento de produto.
Foi realizado apenas um PI, já que houve apenas uma iteração de incremento do produto devido ao tempo de entrega do trabalho.

\section{Desmembrar Features em Histórias de Usuários}
Segundo o SAFe, histórias de usuário são descrições curtas de uma pequena parte da feature a ser implementada.
Nessa tarefa foram desmembradas as features em histórias de usuário, onde cada feature está detalhada com suas histórias abaixo.

\subsection{Histórias de Usuário da Feature: Manter produtos}


\subsection{Histórias de Usuário da Feature: Analisar taxa de entrada/saída de produtos específicos}


\subsection{Histórias de Usuário da Feature: Analisar o uso de matéria-prima}


\subsection{Histórias de Usuário da Feature: Informar sobre os resultados mensais}


\section{Planejar Sprint}
Essa atividade consistiu em planejar a sprint a ser iniciada em 05/11 até 17/11, onde foram selecionadas todas as histórias de usuário para serem implementadas, devido ao tempo apertado.

\section{Desenvolver Histórias de Usuários}

\section{Realizar revisão da iteração}
