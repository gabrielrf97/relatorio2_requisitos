\chapter{Processo de Engenharia de Requisitos}

Na elaboração da primeira etapa do trabalho, o grupo construiu um processo com atividades relacionadas a Engenharia de Requisitos, esse processo está representado abaixo, o capítulo irá abordar os detalhes dessas atividades realizadas de uma forma otimizada.

\begin{figure}[!htpb]
\centering
\includegraphics[scale=0.35]{figuras/processo/modelo}
\caption{Modelo do processo a ser utilizado}
\end{figure}

\newpage

\section{Levantar temas estratégicos}

Segundo o SAFe, temas estratégicos são especicamente objetivos de negócio que conectam o nível de Portfólio com a estratégia de negócio da empresa, no caso do trabalho a equipe, então no contexto do trabalho ao se reunir com o cliente e entender o problema do mesmo, a equipe definiu um único tema estratégico que seria a Gerência de Estoque de forma Adequada.

\begin{itemize}
\item Gerência de Estoque de forma adequada: A proposta do grupo envolve fornecer um produto onde seria demandado de forma tecnológica e prática a gerência do estoque utilizando de uma aplicação para Desktop. Seriam também feitas medidas que ajudassem no controle de desperdício de estoque de produção, para que não haja gastos em excesso.
\end{itemize}

\section{Definir épicos}
Segundo o SAFe, épicos de negócio agregam valor de forma direta no projeto. Foram coletados dois épicos de negócio junto ao cliente, que são listados abaixo:


\subsection{Informatizar Sistema de Controle de Estoque}

	\begin{table}[htb]
    \centering
    \begin{tabular}{|l|l|}
        \hline
        Para          & {\parbox{12cm}{A empresa e seus funcionários}}                                        \\ \hline
        Quem          & {\parbox{12cm}{controlam estoque      }}                                                \\ \hline
        O             & {\parbox{12cm}{Pydaria}}                                                                \\
        É             & {\parbox{12cm}{uma ferramenta para controlar estoque         }}                         \\ \hline
        Que           & {\parbox{12cm}{ajuda a controlar o estoque de uma maneira mais organizada e eficiente.  }} \\ \hline
        Diferente de  & {\parbox{12cm}{ Microsoft Excel       }}                                                 \\ \hline
        Nossa Solução & {\parbox{12cm}{faz esse controle de uma forma mais dinâmica.     }}                     \\
        \hline
    \end{tabular}
    \caption{Posicionamento do épico 1.}
    \end{table}

    \begin{table}[htb]
    \centering
    \begin{tabular}{|l|l|}
        \hline
        Critério de Aceitação & {\parbox{12cm}{O produto tem que fornecer o controle de estoque de uma maneira simples e efetiva, de forma a facilitar o trabalho dos funcionários.}} \\ \hline
        No Escopo             & {\parbox{12cm}{Contabilização de estoque através dos produtos disponíveis para venda e produção. }} \\ \hline
        Fora do escopo        & {\parbox{12cm}{Sincronizar o sistema com algum código identificador nos produtos para identificar sua saída. }}                                       \\
        \hline
    \end{tabular}
    \caption{Escopo do épico 1.}
	\end{table}

\newpage

\subsection{Informatizar sistema para controle de desperdício}
	\begin{table}[htb]
    \centering
    \begin{tabular}{|l|l|}
        \hline
        Para          & {\parbox{12cm}{A empresa e seus funcionários}} \\ \hline
        Quem          & {\parbox{12cm}{controlam estoque}}                                                      \\ \hline
        O             & {\parbox{12cm}{Pydaria}}                                                               \\ \hline
        É             & {\parbox{12cm}{uma ferramenta que dentre outras coisas auxilia os funcionários e proprietários da empresa controlar estoque}}                                \\ \hline
        Que           & {\parbox{12cm}{auxilia o controle de desperdício indicando a melhor tomada de decisão diante de reposição de estoque e produção.}} \\ \hline
        Diferente de  & {\parbox{12cm}{Microsoft Excel e controlar de forma empírica  }}                                                    \\ \hline
        Nossa Solução & {\parbox{12cm}{sugere a tomada de decisão de quantidade de produtos para a reposição ou quantidade de ingredientes para a produção de algum produto dependendo do treinamento de aprendizado de máquina baseado nos dados coletados do tempo do sistema em produção.}}                          \\
        \hline
    \end{tabular}
    \caption{Posicionamento do épico 2.}
    \end{table}


     \begin{table}[htb]
     \centering
    \begin{tabular}{|l|l|}
        \hline
        Critério de Aceitação & {\parbox{12cm}{O software deve sugerir, baseado em dados anteriores registrados pelos envolvidos no controle de estoque e na produção, a quantidade de produtos na reposição de estoque e a quantidade de matéria prima envolvida na produção.}} \\ \hline
        No Escopo             & {\parbox{12cm}{Sugestão de quantidade de produtos para a reposição e matéria prima para produção}}                                                \\ \hline
        Fora do escopo        & {\parbox{12cm}{Sugestões baseadas em outros dados que não seja tempo.}}                             \\
        \hline
    \end{tabular}
    \caption{Escopo do épico 2.}
	\end{table}


\section{Definir enablers dos épicos}

\section{Priorizar Épicos}
Foram levantados apenas dois épicos de negócio, e ambos foram priorizados para serem trabalhados.

\section{Definir Features e Enablers}
Features é um serviço fornecido pelo sistema, de acordo com o SAFe. Seguindo o modelo de rastreabilidade as features precisam ter seus épicos, logo serão detalhadas as features de cada épico separadamente.

\subsection{Features do Épico: Informatizar Sistema de Controle de Estoque}

\begin{itemize}
\item Feature 01: Manter produtos
\begin{itemize}
\item Descrição: Essa feature envolve as operações de CRUD(Create, Read, Update, Delete) básicas da maioria dos softwares, aplicadas aos recursos gerenciados na empresa.
\item Benefícios: Fornecer melhor catalogação dos produtos, substituindo a forma empírica.
\end{itemize}

\item Feature 02: Sincronizar compras e vendas com o armazenamento
\begin{itemize}
    \item Descrição: Manter o estoque sincronizado com o valor correto de produtos.
    \item Benefícios: Permite o controle sobre as quantidades do estoque de forma automatizada, sem que o usuário precise alterar manualmente.
\end{itemize}


\item Feature 03: Analisar taxa de entrada/saída de produtos específicos
\begin{itemize}
	\item Descrição: Verificar se a taxa de saída corresponde a taxa de entrada.
	\item Benefícios: Permite controle do estoque desses produtos contra atos de furto.
\end{itemize}

\end{itemize}

\subsection{Features do Épico: Informatizar sistema para controle de desperdício}

\begin{itemize}

\item Feature 04: Analisar o uso de matéria-prima
\begin{itemize}
	\item Descrição: Analisar se a quantidade de matéria prima obtida corresponde ao ganho equivalente pela venda
	dos produtos derivados daquela matéria prima.
	\item Benefícios: Fornece a análise sobre possíveis falhas no processo de produção, além de permitir o reconhecimento de furto.

\end{itemize}

\item Feature 05: Informar sobre os resultados mensais
\begin{itemize}
	\item Descrição: Informa ao usuário o quadro mensal de gastos, vendas e a atuais análises correspondentes as F3 e F4.
	\item Benefícios: Permite ao usuário uma análise sobre a andamento geral da padaria, e possibilita a ele tomar decisões menos
	intuitivas e mais quantificadas.
\end{itemize}

\end{itemize}

\section{Construir o Visão}

Segundo o SAFe, o documento de Visão descreve uma visão futurística do produto a ser desenvolvido, refletindo o que os stakeholders e clientes necessitam, dando uma visão geral do que é e do que faz o produto.
O documento de visão do projeto segue em anexo ao relatório.


\section{Definir o Roadmap}
Segundo o SAFe, o Roadmap serve para informar as metas de releases do software em cronogramas aproximados.
O RoadMap segue em anexo ao relatório.

\section{Planejar o PI}
O planejamento do Product Increment consiste no encontro cara-a-cara da equipe para definir quantas iterações serão necessárias para implementar aquele incremento de produto.
Foi realizado apenas um PI, já que houve apenas uma iteração de incremento do produto devido ao tempo de entrega do trabalho.

\section{Desmembrar Features em Histórias de Usuários}
Segundo o SAFe, histórias de usuário são descrições curtas de uma pequena parte da feature a ser implementada.
Nessa tarefa foram desmembradas as features em histórias de usuário, onde cada feature está detalhada com suas histórias abaixo.

\subsection{Histórias de Usuário da Feature 01: Manter produtos}

\begin{itemize}
\item US01 - Eu, como usuário, desejo inserir produtos pertencentes ao estoque para mostrar na aplicação.
    \begin{itemize}
    \item Prioridade: Crítico
    \item Dificuldade: 3
    \item Critério(s) de aceitação:
			\begin{itemize}
				\item O usuário poderá cadastrar produtos na aplicação.
				\item O usuário não poderá cadastrar sem informar dados previamente definidos como necessários.
    	\end{itemize}
		\end{itemize}

\item US02 - Eu, como usuário, desejo listar todos os produtos cadastrados na aplicação.
    \begin{itemize}
    \item Prioridade: Crítico
    \item Dificuldade: 1
    \item Critério(s) de aceitação:
		  \begin{itemize}
			  \item O usuário poderá listar todos os produtos previamente cadastrados.
				\item Ao clicar em um produto na lista, o sistema deve informar os dados daquele produto.
			\end{itemize}
		\end{itemize}

\item US03 - Eu, como usuário, desejo procurar por um produto específico cadastrado na aplicação.
    \begin{itemize}
    \item Prioridade: Crítico
    \item Dificuldade: 1
    \item Critério(s) de aceitação:
		  \begin{itemize}
			  \item O usuário poderá procurar por um produto previamente cadastrado.
				\item O sistema deverá disponibilizar em formato de lista, os resultados condizentes com a busca do usuário, semelhante ao US02.
				\item Caso o produto não seja encontrado, o sistema deverá retornar uma mensagem de erro.
			\end{itemize}
		\end{itemize}

\item US04- Eu, como usuário, desejo modificar dados relacionados ao um produto listado na aplicação.
    \begin{itemize}
    \item Prioridade: Crítico
    \item Dificuldade: 1
    \item Critério(s) de aceitação:
			\begin{itemize}
				\item O usuário poderá modificar dados de todos os produtos previamente cadastrados.
			\end{itemize}
    \end{itemize}

\item US05 - Eu, como usuário, desejo remover produtos que não são mais vendidos no comércio da aplicação.
    \begin{itemize}
    \item Prioridade: Crítico
    \item Dificuldade: 1
    \item Critério(s) de aceitação:
			\begin{itemize}
				\item O usuário poderá excluir qualquer produto previamente cadastrado.
			\end{itemize}
    \end{itemize}

\item US06 - Eu, como usuário, desejo acessar o estoque ,visualizando os itens presentes e sua quantidade .
    \begin{itemize}
    \item Prioridade: Importante
    \item Dificuldade: 1
    \item Critério(s) de aceitação:
			\begin{itemize}
				\item O usuário poderá acessar o estoque dos produtos previamente cadastrados.
			\end{itemize}
    \end{itemize}


\end{itemize}

\subsection{Histórias de Usuário da Feature 02: Sincronizar compras e vendas com o armazenamento}
\begin{itemize}

\item US07 - Eu, como usuário, desejo que, ao realizar uma compra, o estoque seja atualizado .
    \begin{itemize}
    \item Prioridade: Importante
    \item Dificuldade: 3
    \item Critério(s) de aceitação:
			\begin{itemize}
				\item O estoque deverá ser atualizado após o usuário informar quais produtos foram obtidos.
				\item O sistema não atualizará um produto que não está previamente cadastrado.
			\end{itemize}
    \end{itemize}

\item US08 - Eu, como usuário, desejo que, ao realizar uma venda, o estoque seja atualizado.
    \begin{itemize}
    \item Prioridade: Crítico
    \item Dificuldade: 3
    \item Critério(s) de aceitação:
			\begin{itemize}
				\item O sistema deverá atualizar o estoque assim que uma venda for confirmada.
			\end{itemize}
    \end{itemize}

\end{itemize}

\subsection{Histórias de Usuário da Feature 03: Analisar taxa de entrada/saída de produtos específicos}

\begin{itemize}

\item US09 - Eu, como usuário, desejo notificar o acréscimo na quantidade de determinado produto registrado no estoque.
    \begin{itemize}
    \item Prioridade: Crítico
    \item Dificuldade: 5
    \item Critério(s) de aceitação:
        \begin{itemize}
        \item O sistema deverá atualizar a quantidade disponível no estoque do produto de entrada.
        \item O sistema deverá ser configurado para atualizar estoques de alguns produtos que já tem entrada confirmada em determinados dias da semana, como refrigerantes às terça-feiras.
        \item O sistema permitirá o usuário editar essas configurações programadas e as quantidades de entrada.
        \end{itemize}
    \end{itemize}

\end{itemize}

\subsection{Histórias de Usuário da Feature 04: Analisar o uso de matéria-prima}

\begin{itemize}

\item US10 - Eu, como usuário, desejo checar os valores sobre o uso de matéria-prima nos processos de produção do próprio estabelecimento.
    \begin{itemize}
    \item Prioridade: Crítico
    \item Dificuldade: 5
    \item Critério(s) de aceitação:
    \begin{itemize}
    \item O sistema deverá manter informações sobre a matéria-prima e suas quantidades utilizadas em processos manuais do comércio, como exemplo, na fabricação de pão de queijo, informando os materiais e uma expectativa de quanto é usado.
    \end{itemize}
    \end{itemize}


\end{itemize}


\subsection{Histórias de Usuário da Feature 05: Informar sobre os resultados mensais}

\begin{itemize}


\item US011 - Eu , como usuário, desejo ser informado sobre os produtos comprados e não vendidos naquele mês.
    \begin{itemize}
    \item Prioridade: Crítico
    \item Dificuldade: 1
		  \item Critério(s) de aceitação:
			\begin{itemize}
				\item O sistema deverá informar os produtos que foram comprados e não foram vendidos durante aquele mês, e sua quantidade.
			\end{itemize}
    \end{itemize}

\item US012 - Eu, como usuário, desejo ser informado se a quantidade de matéria-prima utilizada foi muito inferior a quantidade comprada naquele mês.
    \begin{itemize}
    \item Prioridade: Crítico
    \item Dificuldade: 3
		  \item Critério(s) de aceitação:
			\begin{itemize}
				\item O sistema deverá informar se a quantidade de matéria-prima comprada foi muito superior a quantidade usada.
			\end{itemize}
    \end{itemize}

		\item US012 - Eu, como usuário, desejo visualizar o lucro obtido naquele mês.
		\begin{itemize}
		\item Prioridade: Importante
		\item Dificuldade: 3
			\item Critério(s) de aceitação:
			\begin{itemize}
				\item O sistema deverá informar o lucro obtido naquele mês, tanto sobre a revenda quanto sobre a produção de produtos.
			\end{itemize}
		\end{itemize}


\end{itemize}


\section{Planejar Sprint}
Essa atividade consistiu em planejar a sprint a ser iniciada em 07/11 até 17/11, onde foram selecionadas as seguintes histórias para serem trabalhadas: US01, US02, US04, US05.

\section{Desenvolver Histórias de Usuários}
Essa seção apresenta o status sobre as histórias desenvolvidas na Sprint inicial.

\begin{itemize}
\item US01 - 100\% implementada.
\item US02 - 100 \% implementada.
\item US03 - 0\% implementada.
\item US04 - 0\% implementada.
\end{itemize}
