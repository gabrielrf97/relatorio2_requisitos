\begin{appendices}
\chapter{Visão}

\section{Finalidade}
	A finalidade deste documento é apresentar uma visão geral do software que propõe a solução do problema que afeta a empresa Padaria da Vila, produzido pela equipe.

\section{Escopo}
	A Padaria da Vila é um comércio voltado para a panificação, mercearia e confeitaria localizada no Setor Militar Urbano de Brasília. Nesta empresa a produção de pães é controlada de maneira empírica, pela experiência de longos anos dos proprietários e funcionários, a maior parte do faturamento é voltado aos produtos de mercearia, porém a panificação tem peso importante nas receitas. Com isso foi proposta uma solução de software para facilitar esse controle de estoque e além disso fornecer controle do desperdício de produtos utilizados em produções do próprio comércio.


\section{Não Escopo}
	Não faz parte do escopo do projeto alterar as atividades já efetuadas na padaria, o software irá apenas auxiliar no controle de uma maneira mais confiável.
	Também está fora do escopo a questão de controle de estoque de produtos terceirizados já que os mesmos quando vencidos são trocados pela empresa fornecedora.

\section{Definições, Acrônimos e Abreviações}

\section{Posicionamento}

\subsection{Relato do problema}
	Na primeira etapa do trabalho, na fase de modelagem a equipe já havia tomado conhecimento dos problemas que afetavam a padaria em relação ao controle de estoque.
	O principal motivo da proposta de solução foi que o controle de estoque é feito de modo empírico, sendo muitas vezes antoado em papéis que podem se perder ou apenas fazendo uso da memória dos responsáveis. Outro motivo do cliente pedir um produto de software, foi o caso relatado sobre furto de estoque por parte de funcionários, com isso os donos sempre tinham que gastar mais em matéria-prima, a qual já estava comprada em quantidades corretas e pronta para uso, mas entrava em falta devido a essas ocorrências.

	\begin{table}[htb]
    \begin{tabular}{|l|l|}
        \hline
        O problema          & {\parbox{12cm}{}}                                        \\ \hline
        Afeta          & {\parbox{12cm}{}}                                                \\ \hline
        Cujo Impacto é            & {\parbox{12cm}{}}                                                                \\ 
        Uma boa Solução Seria             & {\parbox{12cm}{}}                         \\ \hline
    \end{tabular}
    \caption{Detalhamento do problema.}
    \end{table}


\subsection{Posicionamento do produto}

	\begin{table}[htb]
    \begin{tabular}{|l|l|}
        \hline
        Para          & {\parbox{12cm}{}}                                        \\ \hline
        Que          & {\parbox{12cm}{}}                                                \\ \hline
        O produto            & {\parbox{12cm}{}}                                                                \\ 
        que             & {\parbox{12cm}{}}                         \\ \hline
        Diferente de             & {\parbox{12cm}{}}                         \\ \hline
        Nosso produto             & {\parbox{12cm}{}}                         \\ \hline
    \end{tabular}
    \caption{Detalhamento do problema.}
    \end{table}

\section{Descrição dos envolvidos e usuários}

\subsection{Resumo dos Envolvidos}

\subsection{Resumo dos Usuários}

\subsection{Principais necessidades dos Usuários ou Envolvidos}

\section{Visão geral do produto}

\section{Características do produto}

\subsection{Interfaces do Produto}

\subsection{Resumo das Capacidades}

\section{Restrições}

\section{Escopo da qualidade}
\end{appendices}