\begin{appendices}
\chapter{Visão}

\section{Finalidade}
	A finalidade deste documento é apresentar uma visão geral do software que propõe a solução do problema que afeta a empresa Padaria da Vila, produzido pela equipe.

\section{Escopo}
	A Padaria da Vila é um comércio voltado para a panificação, mercearia e confeitaria localizada no Setor Militar Urbano de Brasília. Nesta empresa a produção de pães é controlada de maneira empírica, pela experiência de longos anos dos proprietários e funcionários, a maior parte do faturamento é voltado aos produtos de mercearia, porém a panificação tem peso importante nas receitas. Com isso foi proposta uma solução de software para facilitar esse controle de estoque e além disso fornecer controle do desperdício de produtos utilizados em produções do próprio comércio.


\section{Não Escopo}
	Não faz parte do escopo do projeto alterar as atividades já efetuadas na padaria, o software irá apenas auxiliar no controle de uma maneira mais confiável.
	Também está fora do escopo a questão de controle de estoque de produtos terceirizados já que os mesmos quando vencidos são trocados pela empresa fornecedora.

\section{Definições, Acrônimos e Abreviações}
	CRUD - Create, Read, Update, Delete

\section{Posicionamento}

\subsection{Relato do problema}
	Na primeira etapa do trabalho, na fase de modelagem a equipe já havia tomado conhecimento dos problemas que afetavam a padaria em relação ao controle de estoque.
	O principal motivo da proposta de solução foi que o controle de estoque é feito de modo empírico, sendo muitas vezes antoado em papéis que podem se perder ou apenas fazendo uso da memória dos responsáveis. Outro motivo do cliente pedir um produto de software, foi o caso relatado sobre furto de estoque por parte de funcionários, com isso os donos sempre tinham que gastar mais em matéria-prima, a qual já estava comprada em quantidades corretas e pronta para uso, mas entrava em falta devido a essas ocorrências.

	\begin{table}[htb]
	\centering
    \begin{tabular}{|l|l|}
        \hline
        O problema          & {\parbox{12cm}{}}                                        \\ \hline
        Afeta          & {\parbox{12cm}{}}                                                \\ \hline
        Cujo Impacto é            & {\parbox{12cm}{}}                                                                \\ 
        Uma boa Solução Seria             & {\parbox{12cm}{}}                         \\ \hline
    \end{tabular}
    \caption{Detalhamento do problema.}
    \end{table}


\subsection{Posicionamento do produto}

	\begin{table}[htb]
	\centering
    \begin{tabular}{|l|l|}
        \hline
        Para          & {\parbox{12cm}{}}                                        \\ \hline
        Que          & {\parbox{12cm}{}}                                                \\ \hline
        O produto            & {\parbox{12cm}{}}                                                                \\ 
        que             & {\parbox{12cm}{}}                         \\ \hline
        Diferente de             & {\parbox{12cm}{}}                         \\ \hline
        Nosso produto             & {\parbox{12cm}{}}                         \\ \hline
    \end{tabular}
    \caption{Detalhamento do problema.}
    \end{table}

\section{Descrição dos envolvidos e usuários}

\subsection{Resumo dos Envolvidos}

\begin{itemize}

\item Cliente: A representação do papel de cliente se dá pela empresa Padaria da Vila, que é um comércio voltado para atividades de panificação, confeitaria e mercearia, tendo como seu representante nas reuniões a Srta. Thaynara Espindola, para ajudar no planejamento do sistema para solucinar o problema de controle de estoque.

\item Usuários: Os usuários dos sistemas serão aqueles que controlam o estoque, os mesmos terão o papel de fornecer a demanda de produtos para registrar na aplicação e poderão solicitar relatórios da mesma.

\item Equipe: A equipe de desenvolvimento terá participação no projeto, desde o entendimento do problema, levantamento de requisitos à construção do software. A mesma trabalhará com o processo apresentado na primeira etapa do trabalho, utilizando de uma metodologia ágil.

\end{itemize}

\subsection{Principais necessidades dos Usuários ou Envolvidos}

\begin{itemize}

\item Controle efetivo e confiável de estoque
	
	\begin{itemize}
		\item Prioridade: Alta
		\item Problemas:
			\begin{itemize}
				\item O método atual não é tão confiável, podendo haver perda do conhecimento sobre o estoque.
			\end{itemize}
		\item Solução Atual: Forma empírica com uso de papéis ou apenas memorização.
		\item Solução Proposta:
			\begin{itemize}
				\item Cadastrar os produtos em uma plataforma para controlar a quantidade de entrada dos produtos e a atual disponibilidade.
			\end{itemize}
	\end{itemize}

\item Controle de desperdício de matéria-prima

	\begin{itemize}
		\item Prioridade: Alta
		\item Problemas:
			\begin{itemize}
			\item Compra de material em excesso, pode gerar desperdício.
			\end{itemize}
		\item Solução Atual: Compra-se uma quantidade fixa de matéria-prima:
		\item Solução Proposta:
			\begin{itemize}
				\item Analisar mensalmente, os produtos que sobram e que poderiam ser comprado em menor quantidade.
			\end{itemize}
	\end{itemize}

\end{itemize}

\section{Visão geral do produto}

\subsection{Interfaces do Produto}
	A proposta da aplicação Pydaria é uma aplicação Desktop, com uso de uma GUI(Graphical User Interface) produzida em Python com a tecnologia Pyqt4, fornecendo uma interface fácil para utilização dos funcionários da Padaria da Vila, a fim de fornecer detalhes sobre o estoque.

\subsection{Resumo das Capacidades}
	\begin{itemize}
	\item Benefícios para o usuário: Controle confiável de estoque
		\begin{itemize}
			\item Recursos do sistema: O sistema manterá o registro dos produtos no estoque e suas quantidades.
		\end{itemize}

	\item Benefícios para o usuário: Redução de desperdício
		\begin{itemize}
			\item Recursos do sistema: O sistema notificará o usuário sobre matéria-prima comprada em excesso.
		\end{itemize}

	\end{itemize}


\section{Características do produto}

\subsection{Manter Produtos}
	O software permitirá operações de CRUD envolvendo produtos pertencentes ao estoque do comércio, armazenando em um banco de dados.

\subsection{Sincronizar vendas com armazenamento}
	O software utilizará do ato de venda de produtos para atualizar o estoque de acordo com a saída, para tornar o processo mais automático.

\subsection{Analisar entrada/saída de produtos terceirizados}
	O software permitirá os seus usuários verificarem a quantidade de cada produto presente no estoque e também permitirá verificar qual os produtos terceirizados mais vendidos e menos vendidos através de um relatório.

\subsection{Analisar taxa/saída de matéria-prima}
	O software fornecerá uma análise para verificar se a taxa de saída é compatível com a taxa de entrada, a fim de verificar se houve desperdício no mês.

\subsection{Fornecer Relatório}
	O software poderá fornecer um relatório mensalmente ao usuário, informando de algumas possíveis atitudes para evitar o desperdício de recursos.

\section{Restrições}
	O projeto possui as seguintes restrições:
	\begin{enumerate}
	\item A aplicação será executada em desktop, necessitando de uma máquina com sistema operacional Linux ou Windows.

	\item Apenas uma amostra da aplicação será contruído, devido ao tempo disponível para produção.
	\end{enumerate}


\section{Escopo da qualidade}
A análise do escopo da qualidade será efetuada através da verificação da experiência de usuário no uso do produto, ocasionando em manutenções para melhorar a usabilidade caso o cliente solicite.
O produto tem que cumprir o que o usuário solicitou no seu planejamento, caso falte algo o mesmo deve voltar para a produção.


\end{appendices}