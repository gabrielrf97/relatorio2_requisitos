\chapter{Contexto do Cliente}

Nesta seção será descrito o contexto da empresa Padaria da Vila, descrevendo uma visão geral acerca do funcionamento e necessidade da empresa relacionado as suas atividades gerais.

\subsection{Descrição do contexto da empresa Padaria da Vila}

Padaria da Vila é um comércio voltado para a panificação, mercearia e confeitaria localizada no Setor Militar Urbano de Brasília. Nesta empresa a produção de pães é controlada de maneira empírica, pela experiência de longos anos dos proprietários e funcionários, a maior parte do faturamento é voltado aos produtos de mercearia, porém a panificação tem peso importante nas receitas.

Os proprietários são responsáveis por atividades cotidianas, como controle de estoque, controle de fluxo de caixa, gestão de funcionários e inclusive controle de desperdício, atividade que foi notada com problemas em relação a panificação, especialmente com produtos que são  para a fabricação de pães.

A maior parte das atividades por serem realizadas baseadas em conhecimentos adquiridos com a experiência no ramo fogem do controle quanto a relação demanda x oferta, assim sendo ocorrem situações por exemplo de determinado produto acabar antes do dia destinado a realizar pedidos novamente daquele mesmo produto, atrasando e bagunçando o controle de estoque, faltando produtos para a venda e denegrindo a imagem do comércio perante o consumidor, que em outra oportunidades pode preferir ir a outro lugar onde tenha certeza que encontrará o que deseja.

Outro ponto que foge ao controle é quanto a determinação da quantidade necessária de matéria prima para os produtos que são fabricados na padaria. Os donos sabem a quantidade de matéria prima a ser solicitada para a fabricação por semana e sabem que essa quantidade varia de acordo com a época do mês, época do ano e casos especiais como feriados e eventos próximos ao local por exemplo, porém não sabem a relação exata de quanta matéria prima demanda cada unidade de produto e quantas unidades ao certo precisam ser fabricadas para o dia em específico.

\chapter{Análise do problema}

Os problemas foram identificados principalmente através da técnica da entrevista e da observação das atividades cotidianas observadas através de uma visita ao local, o último deu ao grupo ua maior visibilidade causa-consequência dos problemas identificados anteriormente em entrevistas.

Para isto foram elaborados diagramas \textit{fishbones} dos problemas identificados identificando causa para estes e consequência que contribui em relação ao problema abordado.
Para o elaborar o processo é importante que o problema seja conhecido e consiga ser identificado pelos {\itshape Stakeholders} e que a equipe envolvida tenha uma definição comum acerda do problema. A análise do problema é etapa base e fundamental para que os outros processos envolvidos sejam executados, e que o modelo de processo seja condizente com a elaboração do que se refere a soulção.

Foi identificado problemas no processo empíricos envolvidos no controle do estoque e de desperdício, visto que existem situações que não podem ser controladas ou previstas diante de um processo inteiramente humano. Os problemas contidos no processo em questão poderiam ser resolvidos com uma tecnologia apropriada, além de solucionar os defeitos de processos envolvidos, impacta em melhorias gerando menos esforço dos funcionários e mais eficiência no resultado.

\section{Problemas identificados}
