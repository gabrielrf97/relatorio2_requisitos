\chapter{Introdução}

Este trabalho consiste na elaboração do relatório detalhado sobre o processo de engenharia de requisitos executado na disciplina de Requisitos de Software. O contexto abordado envolve a empresa Padaria da Vila, situada no Setor Militar Urbano de Brasília. O processo foi desenvolvido seguindo o modelo de maturidade do CMMI e as atividades contidas no processo estão de acordo com o Scaled Agile Framework (SAFe) e logicamente seguindo a abordagem da metodologia ágil.

\section{Tópicos abordados}

\begin{itemize}
	\item \textit{Capítulo 2 - Contexto do cliente:} Descrição breve sobre as características da empresa, atividades cotidianas realizadas e problemas a serem abordados. 
	\item \textit{Capítulo 3 - Análise do problema:} Detalhamento dos problemas identificados e solução possível para cada um.

	\item \textit{Capítulo 4 - Processo de Engenharia de Requisitos:} Abordagem do processo construído e detalhamento das atividades de cada nível.

	\item \textit{Capítulo 5 - Técnicas de Elicitação:} Feedback sobre o uso das técnicas descritas no trabalho 1.


	\item \textit{Capítulo 6 - Rastreabilidade de requisitos:} Detalhamento sobre a rastreabilidade entre os requisitos levantados.


	\item \textit{Capítulo 7 - Relatos:} Relatos de experiência sobre execução do trabalho e da disciplina.
\end{itemize}
