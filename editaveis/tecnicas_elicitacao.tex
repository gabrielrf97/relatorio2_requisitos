\chapter{Técnicas de elicitação}

Para a elicitação dos requisitos, a princípio, foram propostas cinco técnicas para que os requisitos fossem levantados corretamente, possibilitando assim um desenvolvimento de forma a atender as necessidades do cliente. São elas entrevista orientada, entrevista aberta, observação do ambiente de trabalho, criação de cenário do programa e uso do programa feito pelo usuário.
Por motivos de logistica, não tivemos a oportunidade de botar a técnica de obsevação do ambiente de trabalho em prática. 

A técnica de uso do programa por parte de usuário não pôde ser viabilizada por não termos um programa base além de operações de CRUD para que o nosso usuário pudesse utilizar. Diante dessas baixas, foi colocado foco nas outras técnicas para compensar e manter o trabalho no mesmo nível esperado.

As técnicas que foram de fato aplicadas foram a de criação de cenário de programa e as de entrevista, tanto a orientada onde criamos um roteiro de perguntas a serem feitas ao nosso cliente, quanto a aberta onde tivemos uma conversa sem perguntas previamente definidas, mas que teve foco no problema a ser solucionado. Essas entrevistas resultaram em anotações e um esclarecimento aos membros do time, o que foi fundamental na elicitação dos requisitos com a clareza necessária.